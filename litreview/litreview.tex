%
%  untitled
%
%  Created by Sam Anzaroot on 2012-10-05.
%  Copyright (c) 2012 __MyCompanyName__. All rights reserved.
%
\documentclass[]{article}

% Use utf-8 encoding for foreign characters
\usepackage[utf8]{inputenc}

% Setup for fullpage use
\usepackage{fullpage}

% Uncomment some of the following if you use the features
%
% Running Headers and footers
%\usepackage{fancyhdr}

% Multipart figures
%\usepackage{subfigure}

% More symbols
%\usepackage{amsmath}
%\usepackage{amssymb}
%\usepackage{latexsym}

% Surround parts of graphics with box
%\usepackage{boxedminipage}

% Package for including code in the document
\usepackage{listings}

% If you want to generate a toc for each chapter (use with book)
%\usepackage{minitoc}

% This is now the recommended way for checking for PDFLaTeX:
\usepackage{ifpdf}

%\newif\ifpdf
%\ifx\pdfoutput\undefined
%\pdffalse % we are not running PDFLaTeX
%\else
%\pdfoutput=1 % we are running PDFLaTeX
%\pdftrue
%\fi

\ifpdf
\usepackage[pdftex]{graphicx}
\else
\usepackage{graphicx}
\fi
\title{Database Project Literature Review: Author Linkage in Records using Temporal Information and Conditional Random Fields}
\author{Sam Anzaroot, Jiaping Zheng}

\date{2012-10-17}

\begin{document}

\ifpdf
\DeclareGraphicsExtensions{.pdf, .jpg, .tif}
\else
\DeclareGraphicsExtensions{.eps, .jpg}
\fi

\maketitle

\section{Related Research Areas} % (fold)
\label{sec:related_research_areas}

% section related_research_areas (end)

\section{Sub-Areas} % (fold)
\label{sec:sub_areas}

% section sub_areas (end)

\section{Appropriate papers} % (fold)
\label{sec:appropriate_papers}
One approach to record linking has been to create a hierarchical graphical model (cite w.cohen). This is an improvement on the probabistic approach that builds a classifier over pairwise records in the databse. In this model, there exist latent variables that determine the matching between each field in the record which have a parent variable that determines the matching between the entire record. In this paper, after building the graphical model, they train it as a generative model using an unsupervised manner. The hierarchical graphical model approach is similar to our own approach, but instead of the middle layer containing variables about their own matching, they containing variables dictating the compatibility of their values between the two records in that field that the author is the same. 
Contains connections and differences between them, and relationship to our work.
% section appropriate_papers (end)

\section{Why previous is insufficient} % (fold)
\label{sec:why_previous_is_insufficient}

% section why_previous_is_insufficient (end)

\bibliographystyle{plain}
\bibliography{refs}
\end{document}
