%
%  untitled
%
%  Created by Sam Anzaroot on 2012-10-05.
%  Copyright (c) 2012 __MyCompanyName__. All rights reserved.
%
\documentclass[]{article}

% Use utf-8 encoding for foreign characters
\usepackage[utf8]{inputenc}

% Setup for fullpage use
\usepackage{fullpage}

% Uncomment some of the following if you use the features
%
% Running Headers and footers
%\usepackage{fancyhdr}

% Multipart figures
%\usepackage{subfigure}

% More symbols
%\usepackage{amsmath}
%\usepackage{amssymb}
%\usepackage{latexsym}

% Surround parts of graphics with box
%\usepackage{boxedminipage}

% Package for including code in the document
\usepackage{listings}

% If you want to generate a toc for each chapter (use with book)
%\usepackage{minitoc}

% This is now the recommended way for checking for PDFLaTeX:
\usepackage{ifpdf}

%\newif\ifpdf
%\ifx\pdfoutput\undefined
%\pdffalse % we are not running PDFLaTeX
%\else
%\pdfoutput=1 % we are running PDFLaTeX
%\pdftrue
%\fi

\ifpdf
\usepackage[pdftex]{graphicx}
\else
\usepackage{graphicx}
\fi
\title{Database Project Literature Review: Author Linkage in Records using Temporal Information and Conditional Random Fields}
\author{Sam Anzaroot, Jiaping Zheng}

\date{2012-10-17}

\begin{document}

\ifpdf
\DeclareGraphicsExtensions{.pdf, .jpg, .tif}
\else
\DeclareGraphicsExtensions{.eps, .jpg}
\fi

\maketitle

\section{Related Research Areas} % (fold)
\label{sec:related_research_areas}

% section related_research_areas (end)

\section{Sub-Areas} % (fold)
\label{sec:sub_areas}

% section sub_areas (end)

\section{Appropriate papers} % (fold)
\label{sec:appropriate_papers}
In ``Frameworks for entity matching: A comparison'' the authors decribe many frameworks for entity matching. The main components of an entity resolver are the blocking methods, which pre-partition the records to ensure feasiblity of matching large amount of records. This is either done manually, or semi-automatically. In addition, a record-linkage system employes a matcher, which given attribute value, and/or context information can determine if there is a match. Most work by using similarity metrics, which are either combined and given a threshold, or have learned parameters. Such are numerical approaches, that combined with weights on similarity metrics. Other matchers use rule-based approaches, which also apply and combine similarity functions with thesholds. 

Some of the frameworks mentioned in the above paper treat matchings as black boxes and work on top of the matcher to provide efficient matchings of all the dataset using a matching closure. One such framework is the Swoosh one. In this framework, given that the cononical entity we create during a merge of two records can create further mergings of other records to the same entity, it may be difficult to design a system to effectivly disambiguate an entire database of records given that it is hard or impossible to create transitive matchers. Given that four properties can be linked to a merge, the authors of the paper showed that they can design an efficient system for merging. Since our work focuses on matchers and not a matching system we do not propose a better system than this framework, although we may test against a simple transitive closure. 

One approach to record linking matching has been to create a hierarchical graphical model (cite w.cohen). This is an improvement on the standard method probabistic approach that builds a classifier over pairwise records in the database. In this model, there exist latent variables that determine the matching between each field in the record which have each have a shared parent variable that determines the matching between the entire record. In this paper, after building the graphical model, they train it as a generative model in an unsupervised manner. The hierarchical graphical model approach is similar to our own approach, but instead of the middle layer containing variables about their own matching, they containing variables dictating the compatibility of their values between the two records given that the author of the publication is the same. 
Contains connections and differences between them, and relationship to our work.
% section appropriate_papers (end)

\section{Why previous is insufficient} % (fold)
\label{sec:why_previous_is_insufficient}

% section why_previous_is_insufficient (end)

\bibliographystyle{plain}
\bibliography{refs}
\end{document}
