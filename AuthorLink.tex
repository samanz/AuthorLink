%
%  untitled
%
%  Created by Sam Anzaroot on 2012-10-05.
%  Copyright (c) 2012 __MyCompanyName__. All rights reserved.
%
\documentclass[]{article}

% Use utf-8 encoding for foreign characters
\usepackage[utf8]{inputenc}

% Setup for fullpage use
\usepackage{fullpage}

% Uncomment some of the following if you use the features
%
% Running Headers and footers
%\usepackage{fancyhdr}

% Multipart figures
%\usepackage{subfigure}

% More symbols
%\usepackage{amsmath}
%\usepackage{amssymb}
%\usepackage{latexsym}

% Surround parts of graphics with box
%\usepackage{boxedminipage}

% Package for including code in the document
\usepackage{listings}

% If you want to generate a toc for each chapter (use with book)
%\usepackage{minitoc}

% This is now the recommended way for checking for PDFLaTeX:
\usepackage{ifpdf}

%\newif\ifpdf
%\ifx\pdfoutput\undefined
%\pdffalse % we are not running PDFLaTeX
%\else
%\pdfoutput=1 % we are running PDFLaTeX
%\pdftrue
%\fi

\ifpdf
\usepackage[pdftex]{graphicx}
\else
\usepackage{graphicx}
\fi
\title{Database Project Proposal: Author Linkage in Records using Temporal Information and Conditional Random Fields}
\author{Sam Anzaroot, Jiaping Zheng}

\date{2012-10-05}

\begin{document}

\ifpdf
\DeclareGraphicsExtensions{.pdf, .jpg, .tif}
\else
\DeclareGraphicsExtensions{.eps, .jpg}
\fi

\maketitle

\section{Motivation} % (fold)
\label{sec:motivation}
Record Linking is the task of clustering records in a database such that elements in the same cluster refer to the same real world object. For example, in the DBLP page for David Smith, there are at least three different individuals with publications named David Smith appearing, and the output of a record linker would cluster each David Smith's publications into separate clusters. Various attempts have been made for record linking authors of publications listed in datasets such as DBLP. This task is important because there are many downstream applications that require clean clustering of such information, for example, determining the most prolific authors is only accurate if such clusterings are valid. Network analysis of coauthorship also depends on correct clustering of authors. This task is not a solved one, and we will attempt to build methods that achieve better results.
% section motivation (end)

\section{Methodology} % (fold)
\label{sec:methodology}
First, we will build a dataset based on the DBLP dataset that we will annotate to include accurate information on the author clustering. This will contain both training and testing data. We will also extract author affiliation information from either their homepage or their papers.
This will be done by October 20th.

We will then design a conditional random field model in which the latent variables include the pairwise decision whether authors of two publications are the same. To make the network size manageable, only authors that are spelled similarly are candidates for the pairwise decision variables. This will lead to a clustering by using the transitive closure of the pairwise decisions when inference on the CRF is done. We intend that this CRF also contains latent variables for the compatibility of the other subfields (such as venue or journal) in the publication given that the publication was authored by the same individual. This model is analogous to the CRF that was used to perform record linking over publications on the CORA dataset \cite{Domingos04multi}.
This will be done by Nov 10th.

We will design a number of similarity functions as observations in the CRF. These will include the similarities between titles, venues, and affiliations with the addition of temporal information in the latter. It has been shown that including temporal information specifically for affiliation information improves performance in authorship clustering for publications \cite{DBLP:journals/fcsc/LiDMS12}. For example, it may be the case that two authors have the same affiliation information, but since much time has elapsed, it is not a good indicator that they are the same author.
This will be done by Nov 20th.

We will test on the dataset using the B3 \cite{Bagga98algorithms} and MUC\cite{Vilain95} scores from the coreference resolution community.  These metrics are adaptations of the precision, recall, and F1 metrics, and are designed to measure how close the clusters generated by a system match those in the gold standard.  Since our author linking task's output is a set of clusters of authors, each of which refers to the same real world person, we can evaluate our system's performance by these metrics
This will be done by Dec 1st.
% section methodology (end)

\section{Resources} % (fold)
\label{sec:resources}
We will be using the DBLP dataset as well as the papers that they link to \cite{ley2002dblp}. We will be using the FACTORIE system in our lab \cite{mccallum2009factorie} to build the CRF model and perform learning and inference.
% section resources (end)

\bibliographystyle{plain}
\bibliography{refs}
\end{document}
